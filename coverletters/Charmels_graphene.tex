\documentclass{letter}
\usepackage[utf8]{inputenc}
\usepackage{setspace}
\usepackage{graphicx}
\usepackage[top=0cm,bottom=2.5cm,left=3.5cm,right=3.5cm]{geometry}

\begin{document}

\signature{\vspace{-1.5cm}Óscar Nájera}           % name for signature
\address{Cap.Rafael Ramos E2-254 Casa \#2 \\
Quito,Ecuador\\
(+593-9) 643-9206\\
najera.oscar@gmail.com} 


%Graduate School in Advanced Condensed Matter Science
\begin{letter}{Andreas Isacsson\\
Associate Professor\\
Condensed Matter Theory\\
CHALMERS UNIVERSITY OF TECHNOLOGY}

\begin{flushleft}
{\large\bf Óscar Andrés Nájera Ocampo}
\end{flushleft}
\hrule height 1pt


\opening{To whom it may concern:}
\onehalfspacing

I am writing to apply for the PhD student position in physics, graphene
membrane dynamics for NEMS applications.
Currently, I have finished my undergraduate research and coursework in
physics at the ``Escuela Politécnica Nacional'' where I worked in the
fields of condensed matter and computational physics. I believe
that my research experience and my commitment to excellence in
this fields make me a strong candidate for joining this position.

During my studies I have developed a great taste for theoretical and
mathematical physics especially in the areas of condensed matter
and statistical mechanics. In my research I have linked those areas into
computational physics. Specifically, I have modeled relaxor
ferroelectrics using a random interaction Ising like model. Under
the guidance of Dr. Luis Lascano, I wrote my thesis, entitled
\textit{Estimation, by computer simulation, of the exchange energy
dispersion between polar nano-regions in
$Pb_xBi_4Ti_{3+x}O_{12+3x}; x=\{2,3\}$ relaxor ferroelectrics}.
Here I developed simulation code using Monte Carlo sampling to numerically
solve my model and simulate the relaxor ferroelectric behavior.
Afterwards, simulated data was fitted to experimental measurements
of the real materials searching for correlation between them.

Your offered PhD position is very attractive to me because it spans
over various branches of physics, which I'm already familiar with and enjoy.
I feel very excited to work as a PhD student in this subject, knowing
in advance that great and very interesting challenges are up to.

Beyond my research, I have been fortunate to obtain a wide range
of teaching experience and time managing skills. As my curriculum
vitae indicates, my teaching roles have included teachers assistant
and laboratory assistant. I managed to conclude my research work
on time and well ahead of my former classmates while performing
this high demanding assistant jobs.

You can contact me via email or at (+593-9) 643-9206 and I welcome
the opportunity of a personal interview. I appreciate
you taking the time to review my application and I look forward to
hearing from you. Again, thank you for your consideration.

\closing{Sincerely yours,\vspace{0.5cm}
\includegraphics[bb=0 0 145 54]{./firmas.png}
 % firmas.png: 604x224 pixel, 300dpi, 5.11x1.90 cm, bb=0 0 145 54
}

\end{letter}
\end{document}
