\documentclass{letter}
\usepackage[utf8]{inputenc}
\usepackage{setspace}
\usepackage{graphicx}
\usepackage[top=0cm,bottom=2.5cm,left=3.5cm,right=3.5cm]{geometry}

\begin{document}

\signature{\vspace{-1.5cm}Óscar Nájera}           % name for signature
\address{Cap.Rafael Ramos E2-254 Casa \#2 \\
Quito,Ecuador\\
(+593-9) 643-9206\\
najera.oscar@gmail.com} 

%NanoAndes
\begin{letter}{Luis Calahorrano Garzón\\
Escuela Politécnica Nacional\\
Ladrón de Guevara E11-253\\
Edif. Ing Civil Piso 1\\
Quito-Ecuador}

\begin{flushleft}
{\large\bf Óscar Andrés Nájera Ocampo}
\end{flushleft}
\hrule height 1pt


\opening{To whom it may concern:}
\onehalfspacing

I am writing to apply for the NanoAndes event.
Currently, I have finished my undergraduate research and coursework in
physics at the ``Escuela Politécnica Nacional'' where I worked in the
fields of condensed matter and computational physics. I believe
that my research experience and my commitment to excellence in
this fields makes me a strong candidate for joining your school.

During my studies I have developed a great taste for theoretical and
mathematical physics especially in the areas of condensed matter
and statistical mechanics. In my research I have linked those areas into
computational physics. Specifically, I have modeled relaxor
ferroelectrics using a random interaction Ising like model. Under
the guidance of Dr. Luis Lascano, I wrote my thesis, entitled
\textit{Estimation, by computer simulation, of the exchange energy
dispersion between polar nano-regions in
$Pb_xBi_4Ti_{3+x}O_{12+3x}; x=\{2,3\}$ relaxor ferroelectrics}.
Here I developed simulation code using Monte Carlo sampling to numerically
solve my model and simulate the relaxor ferroelectric behavior.
Afterwards, simulated data was fitted to experimental measurements
of the real materials searching for correlation between them.

I'm very interested in Condensed Matter Physics and Nano-Technology,
and your school offers very appealing talks and tutorials to my
personal interests and further professional development. I aim to
participate in all your school especially in the tutorials about
nanoparticles simulation.

You can contact me via email or at (+593-9) 643-9206 and I welcome
the opportunity of a personal interview. I appreciate
you taking the time to review my application and I look forward to
hearing from you. Again, thank you for your consideration. 

\closing{Sincerely yours,\vspace{0.5cm}
\includegraphics[bb=0 0 145 54]{./firmas.png}
 % firmas.png: 604x224 pixel, 300dpi, 5.11x1.90 cm, bb=0 0 145 54
}
\end{letter}
\end{document}






